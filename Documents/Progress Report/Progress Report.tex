\documentclass[10pt]{article}

%%%%%%%%%%%%%%%%%%%%%%%%%%%%%%%%%%
% PACKAGES
%%%%%%%%%%%%%%%%%%%%%%%%%%%%%%%%%%
\usepackage{multicol} % multiple columns
\usepackage{fancyhdr} % page number in bottom right
\usepackage{graphicx} % images
\usepackage{float} % float image in columns, gives [H]
\usepackage{amsmath, amssymb} % maths symbols / environments
\usepackage{times} % times font instead of computer modern
\usepackage{IEEEtrantools} % et al referencing
\usepackage{booktabs} % nice replacements for \hline in tables
\usepackage{calc} % calculating textwidth-2cm etc.
\usepackage[none]{hyphenat} % no hyphenation
\usepackage{geometry} % Page Margins
\usepackage{hyperref} % PDF metadata setup.
\usepackage{mathptmx} % Use times font face for maths
\usepackage{caption} % Tighter caption control.
\usepackage{multirow}
\usepackage{subfigure}
\usepackage{bookmark}

\usepackage{lipsum} % Temporary to generate body text

%%%%%%%%%%%%%%%%%%%%%%%%%%%%%%%%%%
% TITLE AND AUTHOR
%%%%%%%%%%%%%%%%%%%%%%%%%%%%%%%%%%
% This info is reused in places, so input it here and it will be updated globally.
\newcommand{\docTitle}{Lab Four Report}
\newcommand{\docAuthor}{Yanzhe Zhang}

% Put the metadata in the PDF output.
\hypersetup{
    unicode=true,
    pdftitle={\docTitle{}},
    pdfauthor={\docAuthor{}}
}

%%%%%%%%%%%%%%%%%%%%%%%%%%%%%%%%%%
% FORMATTING REQUIREMENTS
%%%%%%%%%%%%%%%%%%%%%%%%%%%%%%%%%%
\geometry{top=2.5cm, bottom=2.5cm, left=2cm, right=2cm}
\linespread{1.05} % 1.05x line spacing.
\setlength{\columnsep}{0.7cm} % 0.7cm column spacing.
\setlength{\multicolsep}{0cm}
\setlength{\parskip}{6pt} % 6pt skip between paragraphs
\setlength{\parindent}{0pt}
\newcommand{\figsquish}{\vspace{-5mm}} % Hack to fix poor figure spacing due to [H]


%%%%%%%%%%%%%%%%%%%%%%%%%%%%%%%%%%
% SECTION REQUIREMENTS
%%%%%%%%%%%%%%%%%%%%%%%%%%%%%%%%%%
\usepackage{titlesec}
\titlelabel{\thetitle.\hspace{0.5cm}} % Dot between number and title on sections.
% Format is 10pt, so \large = 12pt, \normalsize=10pt
\titleformat*{\section}{\large\bfseries}
\titlespacing*{\section}{0cm}{4pt}{4pt} % 6pt from \parindent
\titleformat*{\subsection}{\normalsize\bfseries}
\titlespacing*{\subsection}{0cm}{0pt}{0pt} % 6pt from \parindent

% Set up footer.
\pagestyle{fancy}
\fancyhf{}
\renewcommand{\headrulewidth}{0pt}
\rfoot{\thepage} % page number, bottom right of page

%%%%%%%%%%%%%%%%%%%%%%%%%%%%%%%%%%
% DOCUMENT BEGIN
%%%%%%%%%%%%%%%%%%%%%%%%%%%%%%%%%%

\begin{document}

% Pull in the IEEE referencing setup stuff.
\bstctlcite{IEEEexample:BSTcontrol}

%%%%%%%%%%%%%%%%%%%%%%%%%%%%%%%%%%
% HEADER
%%%%%%%%%%%%%%%%%%%%%%%%%%%%%%%%%%
{
    \centering
    % Use size 28 font. 1.05x gives 29.4pt line spacing.
    \fontsize{28pt}{29.4pt} \selectfont
    \docTitle\\
    \vspace{25pt}
    % Name block.
    \fontsize{11pt}{11.55pt}\selectfont
    \docAuthor\\
    \fontsize{10pt}{10.5pt}\selectfont
    \textit{yz25g21@soton.ac.uk} \\
}
\vspace{25pt}

%%%%%%%%%%%%%%%%%%%%%%%%%%%%%%%%%%
% BODY
%%%%%%%%%%%%%%%%%%%%%%%%%%%%%%%%%%

\begin{multicols*}{2}

\section{Radial Basis Function (RBF)}

Figure~\ref*{fig:normalize} shows the result after the input data are normalized to have a mean of 0 and standard deviation of 1.
This is been done by appling: 

\figsquish
\begin{gather*}
    X_{normalized} = \frac{X-\mathrm{mean} (X )}{\mathrm{std} (X)} 
\end{gather*}

The aim of normalization is to bring all features to a similar range, preventing one feature from dominating due to its larger scale. 

\figsquish
\begin{figure}[H]
    \centering
    \includegraphics[width=\columnwidth]{normalize.png}
    \caption{Normalized input data}
    \label{fig:normalize}
\end{figure}

If the selected pair of points happens to be close to each other, the basis functions may have a narrow range. This could result in a model that is too sensitive to local variations, might result in overfit of the training data,
capturing noise and outliers. 
Also, the basis function range based on a random pair of points might not be robust to changes in the dataset.
Figure~\ref*{fig:sigma} shows the result after using the average of four pairs of data points instead of one pair. The randomness has been reduced. 

\figsquish
\begin{figure}[H]
    \centering
    \includegraphics[width=\columnwidth]{sigma.png}
    \caption{Average of Four Pairwise Distances}
    \label{fig:sigma}
\end{figure}

The use of K-means clustering is useful for the datasets which has some degree of clustering structure. Those centers can serve as the locations of the basis function, 
this will ideally resulting a better performence. 

\figsquish
\begin{figure}[H]
    \centering
    \includegraphics[width=\columnwidth]{split.png}
    \caption{Split the data into Training and Testing sets}
    \label{fig:split}
\end{figure}

The performance for the test set is not good compare to the training set. This is a hint of overfiting. 
This could cause by a significant difference in the distribution between the test set and the training set. 
A cross validation is aim to minimize this. 


The MSE of the test set results is plotted in figure~\ref*{fig:cross_validation}, compared with Linear Regression Models.

\figsquish
\begin{figure}[H]
    \centering
    \includegraphics[width=\columnwidth]{cross_validation.png}
    \caption{ten-fold cross validation, linear regression}
    \label{fig:cross_validation}
\end{figure}

\section{Multi-Layer Perceptron (MLP)}

The two classification problem is generated as in figure~\ref*{fig:Classification}. 

\figsquish
\begin{figure}[H]
	\centering
	\subfigure[Easy to Learn]{
	\begin{minipage}[H]{0.45\columnwidth}
	\includegraphics[width=1\columnwidth]{easy_class.png}
	\end{minipage}
	\label{fig:easy_class}
	}
    \subfigure[Hard to Learn (20\% overlap)]{
    \begin{minipage}[H]{0.45\columnwidth}
    \includegraphics[width=1\columnwidth]{hard_class.png}
    \end{minipage}
	\label{fig:hard_class}
    }
	\caption{Classification}
	\label{fig:Classification}
\end{figure}

MLP classifier with hidden\_layer\_sizes = (128, 64), max\_iter = 1000 is used to train the data. 

\figsquish
\begin{figure}[H]
    \centering
    \includegraphics[width=\columnwidth]{Bayesian vs MLP.png}
    \caption{Bayesian vs. MLP}
    \label{fig:Bayesian vs MLP}
\end{figure}

The result of Bayesian classifier and MLP classifier shows that for the easy one, both of them have a good saperation. 
However for the overlapped data, Bayesian is slightly better than MLP with given parameters. 

Ten-fold cross validation was implemented for the MLP classifier. For the first paritition, Figure~\ref*{fig:boundaries} shows a comparesion of Gaussian classifier,
a simple MLP and a complex MLP classifier. 

\figsquish
\begin{figure}[H]
    \centering
    \includegraphics[width=\columnwidth]{boundaries.png}
    \caption{Boundaries of Gaussian; Simple MLP; MLP}
    \label{fig:boundaries}
\end{figure}

The simple MLP uses only 5 hidden layers, is the worst one among three of them. As the number of hidden layer gets large, the performance gets better. 

\figsquish
\begin{table}[H]
    \centering
    \caption{Parameters of MLP}
    \begin{tabular}{|c|c|}
        \hline
        Parameters & Role\\
        \midrule
        hidden\_layer\_sizes & number of neurons in each layer\\
        activation & \{'identity', 'logistic', 'tanh', 'relu'\}\\
        & the activation function used\\
        solver & \{'lbfgs', 'sgd', 'adam'\}\\
        & the solver for weight optimization \\
        alpha & Strength of the L2 regularization term.\\
        batch\_size & Size of minibatches for stochastic \\ & optimizers\\
        learning\_rate & \{'constant', 'invscaling', 'adaptive'\}\\
        learning\_rate\_init & The initial learning rate used. \\
        power\_t & The exponent for inverse scaling \\ & learning rate\\
        max\_iter & Maximum number of iterations.\\
        shuffle & Whether to shuffle samples in \\ & each iteration.\\
        random\_state & Determines random number generation \\ & for weights and bias initialization\\
        tol & Tolerance for the optimization.\\
        verbose & Whether to print progress \\ & messages to stdout.\\
        momentum & Momentum for gradient descent update. \\
        nesterovs\_momentum & Whether to use Nesterov's momentum. \\
        early\_stopping & Whether use early stopping to terminate \\ & training as validation score isn't improving.\\
        beta\_1 & Exponential decay rate for estimates\\ & of first moment vector in adam\\
        beta\_2 & second moment vector in adam\\
        epsilon & Value for numerical stability in adam\\
        \hline
    \end{tabular}
    \label{tab:Parameters}
\end{table}

By altering the learning\_rate\_init, from default 0.001 to 0.1, The convergence decreases more rapidly, and the loss is smaller. 

\figsquish
\begin{figure}[H]
    \centering
    \includegraphics[width=\columnwidth]{lr_distr.png}
    \caption{Boundaries change as lr changes}
    \label{fig:lr_distr}
\end{figure}

\figsquish
\begin{figure}[H]
    \centering
    \includegraphics[width=\columnwidth]{lr_conver.png}
    \caption{Convergence as lr changes}
    \label{fig:lr_conver}
\end{figure}

With changing the hidden\_layer\_sizes to (10, ), which means one layer with 10 neurons, the convergence decreases slower. It consumes more time to train the model, and the performance is not as good. 


\figsquish
\begin{figure}[H]
    \centering
    \includegraphics[width=\columnwidth]{hl_distr.png}
    \caption{Boundaries change as hl changes}
    \label{fig:hl_distr}
\end{figure}

\figsquish
\begin{figure}[H]
    \centering
    \includegraphics[width=\columnwidth]{hl_conver.png}
    \caption{Convergence as hl changes}
    \label{fig:hl_conver}
\end{figure}

\end{multicols*}

\nocite{*}

\bibliographystyle{IEEEtran}
\bibliography{IEEEabrv,mybib.bib}

\end{document}

