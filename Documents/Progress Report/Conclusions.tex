%% ----------------------------------------------------------------
%% Conclusions.tex
%% ---------------------------------------------------------------- 
\chapter{Conclusions \& Proposal} \label{Chapter: Conclusions}

\section{Conclusions}

A Gantt Chart of all the work done is shown in \aref{Appendix:listing}, \fref{Figure:gantt}.
Up to this point, the background of the LSTM model has been well-researched. The dataset has been found, and the data are well understood. 
Several method to preprocess the data in temporal aspects has been go through, to reconstruct a better dataset fed into the model. 
A simple LSTM model has been built, given a basic working skeleton of this project. 

\section{Future Work}

The chart also states the work not completed and planned to be done in the future.

The deep learning library still needs to be more familiarized. It is the priority to have a better structure for the network 
by altering the number of LSTM layers, batch sizes, and so on. These hyperparameters could be optimised using Bayesian optimisation. 

The information on upstream and downstream has not been used for now, consider using spatio-temporal graph convolutional networks (STGCN) method \cite{DBLP:journals/corr/abs-1709-04875}\cite{8560205} to extract 
traffic's spatial characteristics as well, to potentially reach better performance. Hence, further background research on Bayesian optimisation and STGCN is required. 

The LSTM model is sensitive to missing values, consider the LSTM-M model \cite{TIAN2018297} to reduce its impact.

A navigation algorithm based on the A* method will be implemented to show how the model can benefit daily lives. 