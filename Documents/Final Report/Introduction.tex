%% ----------------------------------------------------------------
%% Introduction.tex
%% ---------------------------------------------------------------- 
\chapter{Introduction} \label{Chapter:Introduction}

In modern, fast-paced lives, urban mobility is an integral aspect that significantly impacts our daily routines. 
One of the most pervasive challenges in urban areas is the issue of traffic congestion, causing delays, frustration, and inefficiencies in transportation systems. 
This poses high requirements for traffic management and navigation systems addressing these problems. 
In the topic of traffic management and planning, accurate prediction of vehicles and proactively suggesting optimal routes in urban road networks is one of the important tasks to improve traffic efficiency and safety.
This project seeks to explore the realm of traffic prediction, employing deep learning techniques to provide accurate insights into future traffic conditions. 

The project focuses on near-future predictions, more specifically, predicting the traffic condition in half-hour advance. 
There are two approaches to the aims that are covered, one is to train a model for a single location, predicting traffic at individual points on the map.
This is referred to as the Localised designs in this report. 
By focusing on specific locations within the map, localised predictions offer detailed insights into traffic patterns and congestion levels at particular points of interest. 
It allows for targeted interventions and optimizations at specific locations. It can also provide a source for navigation apps with real-time traffic information and suggest routes to avoid congested areas. 

The other approach is to consider an area of roads as a whole, predicting the traffic for all roads of that area at the same time, namely the Globalised designs. 
It offers a holistic view of traffic conditions across a certain area, enables city authorities to manage traffic flow, and implements politics to alleviate congestion on a broader scale.